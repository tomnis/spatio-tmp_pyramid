
\subsection{Related Work}

%1) Brief para on activity recognition approaches in general - see some of my recent papers to grab those refs and text

For generic (non-egocentric) activity recognition, methods based on tracked limbs and body shapes (e.g., \cite{Ramanan03, Rao01, Rodriguez08}) analyze human actions in a model-based way.  More recently, model-free alternatives based on low-level descriptors of gradients and optical flow have been explored (e.g.,~\cite{Schuldt04,Laptev08,Marszalek09}), attempting to directly learn the motion and appearance patterns associated with an activity.  A fairly standard pipeline has emerged analogous to the bag-of-visual-words approaches often employed for image classification: detect space-time interest points, extract local descriptors for each point, quantize to space-time visual words, then represent the entire video with a histogram counting how often each word appears.

%2) Bag of words methods for video - straight bow, then elaborations...Laptev, Choi, Ramanan (object occurrences for ego, rather than stips), Adriana,...

Since a pure bag-of-words lacks any notion of ordering, researchers have further drawn inspiration from spatial pyramid image representations~\cite{Lazebnik06,Bosch07} to construct space-time histograms from subcells within the video volume.  These subcells count features appearing in particular regions of the video, and as such they can flexibly capture the relative layout.  In~\cite{Laptev08}, a set of spatio-temporal bin structures is defined that uses six possible spatial grids and four temporal binning schemes, resulting in a total of 24 possible spatio-temporal partitions.  The histograms from all partitions are combined by a summed kernel.  In~\cite{Choi08}, a space-time pyramid with a hierarchy of regularly sized cubic bins is constructed, and used to pool the features at multiple resolutions.   A related strategy is to hierarchically bin neighboring local features and discriminatively learn which space-time weightings are most informative~\cite{Kovashka10}.  For the egocentric setting, a temporal pyramid that divides the video into half along the temporal axis (and uses no spatial partitions) is proposed, and used to histogram object detector outputs~\cite{Ramanan12}.  Unlike any of these approaches, our idea is to learn which pyramid structures are discriminative.


%3) egocentric video analysis work - includes recognition and other apps like summarization, social role discovery etc-- cite Yong Jae, Lu Zheng, Fathi/Ren papers here, as well as refs in Yong Jae / Lu Zheng paper about ego work --- conclude that much of the work establishes the importance of *objects* in ego activity rec; they drive the actions (as opposed to say body pose as an important signal in other forms of activity rec, or low-level stips)

Prior work on activity recognition from wearable cameras~\cite{Sundaram,Hanheide,Fathi-ICCV2011} often considers a particular environment of interest (like a particular kitchen) for which individual familiar objects are informative, and some explores the role of additional sensors such as Inertial Measurement Units~\cite{Spriggs}.  In contrast, we are interested in recognizing activities by a camera wearer moving about multiple environments and without additional sensors or pre-placed objects of interest.  Such a setting is also tackled in the recent work of~\cite{Ramanan12}.  We leverage their finding that object-based representations are critical for egocentric activity, and also use their idea of ``active'' objects.  However, while that method uses a simple hand-crafted histogram structure consisting of two temporal bins (one for the first half of the video, one for the second half of the video), we propose to learn a boosted combination of discriminative histogram partitions.  Through direct comparison in our results, we show that our idea achieves substantially more accurate activity recognition.

Aside from recognizing activities, egocentric video analysis also entails interesting problems in object recognition~\cite{ren-gu-cvpr2010}, event segmentation~\cite{clarkson}, novelty detection~\cite{novelty}, summarization or unsupervised discovery~\cite{jojic,kitani,Lee12}, and the relationship between gaze and activity~\cite{Fathi12}.


% add lu zheng


%4) spatial binning strategies and learning thereof - Lazebnik, Sharma, Jiang,...; for this, do not give too much credit to Jiang.  Just say we explore discriminative partition selection for video data, and specifically in ego centric domain...
Our approach to learn discriminative space-time bin structures for activity recognition takes inspiration from methods for image classification that select discriminative spatial bins~\cite{Sharma11,Jiang12}.  In~\cite{Sharma11}, the spatial grid and classifier are jointly learned using a maximum margin formulation, and in~\cite{Jiang12}, boosting is used to select useful randomized spatial partitions.  Both methods target scene classification from images.  In contrast, we learn discriminative partitions in space-time for activity recognition.  To our knowledge, no prior work considers discriminative learning of spatio-temporal partitions, whether for egocentric or non-egocentric data.  Furthermore, our idea to bias the randomized partitions to focus on active objects is novel, and is critical for recognition results, as we will show in experiments.


%%%%%%%%%%%%%%%%%%%%%%%%
%%%%%%%%%%%%%%%%%%%%%%%%



	%
%  Existing work has established that egocentric activities are object-driven
%  in the sense that visible objects provide useful cues about what types of
%  activities are occurring, rather than tracking of limb position or
%  summarization of overall motion. In other words,
%	egocentric activity recognition is ``all about
%  the objects'' \cite{Ramanan12}, particularly the objects being interacted
%  with (``active objects''), as
%	recognition accuracy increases dramatically when locations of active
%  objects in addition to passive objects are used as features.
%

